\newpage
\chapter{SONUÇ}\label{ch:sonuc}
\paragraph{}
Bu çalışmada, nötrinoların kollektif çeşni evriminin, nötrino elektromanyetik alan etkileşimi varlığında nasıl değiştiğini hem analitik olarak hem de sayısal olarak inceledik. Elde ettiğimiz analitik sonuçların ÇÇSN soğuma evresinin erken dönemi için tutarlı ve doğru sonuç verdiğini gördük. ÇÇSN meydana geldikten yaklaşık $ 3-4 $ saniye sonra, MSW ve SFP rezonanslarının gerçekleştiği uzaklıklar yakınlaştığı için iki çeşniye indirgenmiş analitik çözümler ile sayısal çözümler arasında farklılıklar ortaya çıkmaktadır. Faz etkileri ile kendisini gösteren bu farklılık yüksek enerjili nötrinolarda daha belirgin hale gelmektedir.

Yaptığımız çalışmayı özetlersek, öncelikle nötrinoların manyetik alan ve madde etkileşimleri altında çeşni evrimini veren hareket denklemlerini yazdık. Bu denklemleri iki farklı şekilde çözdük. Birincisi, manyetik alanın etkisini tedirgenmiş bir potansiyel olarak alıp evrimin özvektörlerini elde ettik. Bu özvektörlere karşılık gelen özdeğer katkılarını hesapladık. Ardından eksponansiyel olarak değişen baryon yoğunluğu ve sabit manyetik alan altında, iki çeşniye indirgenmiş hareket denklemlerinin çözümünün konfluent hipergeometrik fonksiyonlar olduğunu elde ettik. Ayrıca manyetik alanın baryon yoğunluğu ile aynı eksponansiyel azalmaya sahip olduğunda, çözümlerinin genelleştirilmiş hipergeometrik fonksiyonlar ve birleşmiş Laguerre polinomları cinsinden verildiğini bulduk.

ÇÇSN içerisindeki çeşni evrimini incelemek adına oyuncak modeller kurduk. Bunun için yoğunluk operatörünün özbazdaki analitik ifadesini, LZ geçiş olasılıkları ve fazların katkısını da dahil ederek yazdık. Yoğunluk operatörüne sırayla salınım fazını, SFP rezonansından kaynaklanan LZ geçiş olasılığını ve bunun fazını ve son olarak MSW rezonansından kaynaklanan LZ geçiş olasılığını ve bunun fazını ekledik. Elde ettiğimiz ifadeyi \emph{ortalamadan} sorumlu, \emph{hatadan} sorumlu ve \emph{salınımdan} sorumlu terim  olarak üçe ayırdık. Sonsuz uzaklıkta salınımdan sorumlu terim sıfıra gideceği için bu terimi hesaba katmadık. Tümüyle adyabatik evrim olması durumunda, fazlardan gelen katkıların yani hatadan sorumlu olan terimin sıfır olacağını elde ettik. Hatadan sorumlu kısma gelen tüm katkılar nötrinoların rezonanslara farklı fazlarla girmelerinden kaynaklanır ve bunlar sadece geçişleri adyabatik olmadığında kendini gösterir. Bu etkiyi daha iyi gözlemleyebilmek için adyabatik olmayan ÇÇSN'ya ait başlangıç koşullarında küçük değişiklikler yapıp fazların etkisini ortaya çıkardık. SFP ve MSW rezonansının birbirinden yeteri kadar ayrıldığı çeşni evrimde faz etkileri, analitik öngörülerimzle tam olarak uyumlu çıkmaktadır. Burada bahsettiğimiz "yeteri kadar" kavramını niceliksel olarak tanımladık ve rezonans noktalarındaki $ \sin^{2}(2\theta) $ teriminden yarım uzunluk yarım maksimum değerlerini analitik olarak elde ettik. Yaptığımız analizler sonucunda ÇÇSN meydana geldikten yaklaşık $ 3-4 $ saniye sonra MSW ve SFP rezonansının birbirine yaklaştığını ve üç çeşni etkilerin ortaya çıktığını bulduk. Yoğunluk operatörünün hatadan sorumlu kısmı, başlangıçta Hamiltonyen'in enerji özdurumlarının çeşni özdurumları olmaktan ne kadar uzak olduğuna da bağlıdır. Bu da doğrudan $ \mu B $ teriminin madde potansiyeline göre ne kadar güçlü olduğuna bağlıdır. Buna göre $ \mu $ ya da $ B $ artacak olursa gözlemsel belirsizlik de artacaktır.

Analitik çözümler ile sayısal çözümleri karşılaştırmamızın ardından gerçekçi ÇÇSN modeli için simülasyon yaptık. Elektromanyetik etkileşimlerin ve nötrino öz-kırılımının çeşni evrimine olan etkilerine bakmak için dört ayrı model ile çalıştık. Elektromanyetik etkileşimlerin $ x $ antinötrino yoğunluğunu arttırdığını bulduk. Ayrıca, bu etkileşimin kollektif nötrino salınımlarından kaynaklanan spektral ayrışmayı belirsizleştirdiğini, hatta antinötrino enerji spektrumunda bulunan spektral ayrışmayı yok ettiğini gözlemledik.

Bu tezin son bölümünde ise nötrino işlemi çekirdek sentezlenme sonucunda $ ^{7} $Li ve $ ^{11} $B gibi nötrino-işlem elementlerinin kütle kesirleri elde edilmiştir. Nötrino etkileşimlerinin toplam ürün üzerindeki etkisine bakılmıştır. Çekirdek sentezleme hesapları yapılırken nötrino işlemi dikkate alınırsa Lityum elementi $ 10^{10} $ kat, Bor elementi ise $ 10^{5} $ kat daha bol bulunacaktır.

Tüm nötrino-işlem hesapları, başka ata yıldız modelleri ve onlara uygun süpernova simülasyonları kullanılarak yapılabilir. İstatistik arttırılarak, Güneş sistemindeki element bollukları bu yolla açıklanabilir. Ayrıca iki major nötrino-işlem element üretim mekanizması anlaşılmasının ardından Güneş sistemi ile daha sağlıklı karşılaştırma olanağı bulabiliriz. 

Bu çalışmanın bir sonraki adımı SFP ile MSW rezonansı arasında kalan donmuş fazların çeşni evrimine olan katkısını incelemektir. Elektron kesrinin aniden değiştiği Yıldız'ın iç bölgelerinde nötrino enerji spektrumu, donmuş fazlardan dolayı spektral ayrışmaya benzer bir davranışta bulunacaktır. Ayrıca bu bölge, nötrino öz-kırılım etkilerinin de meydana geldiği bölge olduğu için birden fazla spektral ayrışmanın meydana gelmesi muhtemeldir.