\chapter*{\begin{center}\bfseries\large
KOLLEKTİF EVRİLEN NÖTRİNOLARIN ÇEŞNİ DÖNÜŞÜMÜ\\
(Doktora Tezi)\\
İsmail Taygun BULMUŞ\\
\vspace{0.5cm}
MİMAR SİNAN GÜZEL SANATLAR ÜNİVERSİTESİ\\
FEN BİLİMLERİ ENSTİTÜSÜ\\
Haziran 2022\\
\vspace{0.5cm}
ÖZET
\vspace{0.1cm}
\end{center}}
\addcontentsline{toc}{chapter}{ÖZET}

Nötrinolar evrende var olan diğer parçacıklarla çok küçük tesir kesitleri ile etkileşirler. Buna karşın çekirdek çökmeli süpernova (ÇÇSN) gibi bir çok astrofiziksel fenomeni açıklamada kritik rol oynarlar. Bu tezde ÇÇSN içindeki nötrino çeşni evrimini inceledik ve iki çeşni yaklaşıklığı altında kollektif çeşni evrimini veren analitik sonuçlar elde ettik. 
Madde ortamından geçerken meydana gelen MSW rezonansını ve manyetik alandan kaynaklanan SFP rezonansını detaylıca çalıştık. 
İki seviyeli sistemler için yazılan Landau-Zener (LZ) geçiş olasılıklarını yoğunluk operatörünün evrimine dahil edip sayısal simülasyonlar ile karşılaştırdık. Yoğunluk operatörüne gelen fazları başlangıç koşullarındaki küçük değişiklikler dahilinde inceleyip ortalama yoğunluk operatörünün analitik ifadesini elde ettik. SN1987A ÇÇSN modelinin $t=1-3 $ s arasındaki madde profili dikkate alındığında öngörüler ve sayısal sonuçlar uyumlu çıkmaktadır. $ t=4$ saniyeden sonra, özellikle yüksek enerjilerde, MSW ve SFP rezonansları birbirine yakınlaşır. Bu etki iki çeşni yaklaşıklığının geçerliliğini azaltır. Bu analitik sonuçların geçerlilik koşullarını efektif karışım açılarından ürettik. Ardından şok dalgası dahil edilmiş gerçekçi ÇÇSN modeli kullanıp öz-kırılım etkisini dahil ettiğimizde, antinötrinoların enerji spektrumunda oluşan spektral ayrışmanın kaybolduğunu ve elektron antinötrinolarının ısındığını bulduk. Son olarak nötrinoların ÇÇSN içerisindeki çekirdek sentezlenmesindeki rolüne baktık ve nötrino çekirdek sentezlenmesi dahil edildiğinde lityum ve bor bolluklarının arttığını bulduk. 

\begin{description}\itemsep0.05pt
%\item[Bilim Kodu:] 123
\item[Anahtar Kelimeler:] Nötrino Çeşni Salınımı, LZ geçiş olasılığı, Süpernova. 
\item[Sayfa Adedi:] 131
\item[Tez Yöneticisi:] Prof. Dr. Yamaç PEHLİVAN
\end{description}
%\clearpage

\chapter*{\begin{center}\bfseries\large
FLAVOR TRANSFORMATIONS OF COLLECTIVE EVOLVED NEUTRINOS\\
(Ph.D. Thesis)\\
İsmail Taygun BULMUŞ\\
\vspace{0.5cm}
M\.{I}MAR S\.{I}NAN FINE ARTS UNIVERSITY\\
INSTITUTE OF SCIENCE AND TECHNOLOGY\\
June 2022\\
\vspace{0.5cm}
SUMMARY
\end{center}}
\addcontentsline{toc}{chapter}{SUMMARY}

Neutrinos interact with other particles in the universe with very small cross-sections. However, they play a critical role in explaining many astrophysical phenomena, such as core collapsed supernovae (CCSN). In this thesis, we considered flavor evolution of neutrinos in a CCSN and obtained analytical results that give the collective flavor evolution under the two flavor approximation. We studied the MSW resonance that occurs when passing through the matter and the SFP resonance caused by the magnetic field. We included the Landau-Zener (LZ) transition probabilities which are valid for two-level systems in the evolution of the density operator and compared results with numerical simulations.We obtained the analytical expression of the average density operator by examining the phases of the density operator with small changes in the initial conditions. Considering the baryon profile of the SN1987A CCSN model between $t=1-3 $ s, the predictions and numerical results are compatible. After $ t=4$ second, especially at high energies, the MSW and SFP resonances become closer together. This effect reduces the validity of the two-flavor approximation. We generated the validity conditions for our analytical results in terms of effective mixing. Then, we found that when we include the self-refraction effect using the parametric shock wave included CCSN model, the spectral split in the energy spectrum of the antineutrinos disappeared and the electron antineutrinos got warmer. Finally, we looked at the role of neutrinos in nucleosynthesis in the CCSN. We looked at its role in nucleosynthesis in the CCSN and found that lithium and boron abundances increased when neutrino process is included.

\begin{description}\itemsep0.05pt
\item[Key Words:] Neutrino Flavor Evolution, LZ Transition Probability, Supernovae.
\item[Page Number:] 131
\item[Supervisor:] Prof. Dr. Yamaç PEHLİVAN
\end{description}
%\clearpage
