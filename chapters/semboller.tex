%\renewcommand*{\arraystretch}{1.37}
\chapter{SEMBOL LİSTESİ}
\begin{longtable}{@{}l @{\hspace{10mm}} l }
$ c_{\alpha}            $ &: $\cos \alpha$ \\
$ s_{\alpha}            $ &: $\sin \alpha$ \\
$ r                     $ &: Uzaklık \\
$ E                     $ &: Enerji \\
$ m_{i}                 $ &: Nötrino Kütlesi \\
$ \delta m^{2}          $ &: Nötrino Kütle Kare Farkı \\
$ \Delta                $ &: $\delta m^{2}/4E$ \\
$ \mu                   $ &: Nötrino Dipol Momenti \\
$ \hat{H}^{f}           $ &: Çeşni Tabanında Hamiltonyen Operatörü\\
$ \hat{H}^{k}           $ &: Kütle Tabanında Hamiltonyen Operatörü\\
$ \hat{H}^{M}           $ &: Madde Tabanında Hamiltonyen Operatörü\\
$ P_{\nu_{\alpha}\rightarrow\nu_{\beta}}$ &: $ \alpha $ Çeşnisinden $ \beta $ Geçiş Olasılığı \\
$ P_{\nu_{\alpha}\rightarrow\nu_{\alpha}}$ &: $ \alpha $ Çeşnisine Sahip Olan Nötrinonun Yaşama Olasılığı\\
$ A                     $ &: Geçiş Genliği\\
$ \ket{\Psi(r)}         $ &: Nötrino Durum Keti\\
$ \ket{\nu_{\alpha}}    $ &: Çeşni Tabanında Nötrino Keti, $ \alpha = e,x,\overline{e},\overline{x} $ \\
$ \ket{\nu_{i}}         $ &: Kütle Tabanında Nötrino Keti, $ i = 1,2,3,4 $ \\
$ \ket{\nu_{i}^{M}}     $ &: Madde Tabanında Nötrino Keti, $ i = 1,2,3,4 $ \\
$ \ket{\nu_{i}^{EM}}    $ &: Elektromanyetik Tabanında Nötrino Keti, $ i = 1,2,3,4 $ \\
$ \theta                $ &: Boşluk Karışım Açısı \\
$ \theta_{M}            $ &: Nötrinolar İçin Efektif Madde Karışım Açısı \\
$ \overline{\theta}_{M} $ &: Antinötrinolar İçin Efektif Madde Karışım Açısı \\
$ \gamma                $ &: $\theta_{M}-\overline{\theta}_{M}$ \\
$ \theta_{EM}           $ &: Efektif Elektromanyetik Karışım Açısı \\
$ \mathcal{R}           $ &: Dönme Matrisleri \\
$ U                     $ &: Boşluk Dönüşüm Matrisi \\
$ U_{M}(r)              $ &: Efektif Madde Dönüşüm Matrisi \\
$ U_{EM}(r)             $ &: Efektif Elektromanyetik Dönüşüm Matrisi \\
$ \lambda^{\nu}_{i}     $ &: Boşluk Salınım Hamiltonyen'in Özdeğerleri, $ i = 1,2,3,4 $ \\
$ \lambda^{\nu,EM}_{i}  $ &: Boşluk Salınım ve Elektromanyetik Hamiltonyen Toplamlarının \\
& Özdeğerleri, $ i = 1,2,3,4 $ \\
$ \lambda_{i}           $ &: Boşluk Salınım, Madde ve Elektromanyetik Hamiltonyen Toplamlarının \\
& Özdeğerleri, $ i = 1,2,3,4 $ \\
$ \omega_{i}            $ &: Madde Hamiltonyen'in Özdeğerleri, $ i = 1,2,3,4 $ \\
$ V_{NC}(r)             $ &: Efektif Yüksüz Akım Potansiyeli \\
$ V_{CC}(r)             $ &: Efektif Yüklü Akım Potansiyeli \\
$ G_{f}                 $ &: Fermi Çiftlenim Sabiti \\
$ N_{e}                 $ &: Elektron Yoğunluğu \\
$ N_{n}                 $ &: Nötron Yoğunluğu \\
$ N_{p}                 $ &: Proton Yoğunluğu \\
$ N_{b}                 $ &: Baryon Yoğunluğu \\
$ Y_{e}                 $ &: Elektron Kesri \\
$ Y_{n}                 $ &: Nötron Kesri \\
$ B                     $ &: Manyetik Alan \\
$ P_{LZ}                $ &: Landau - Zener Geçiş Olasılığı \\
$ \Gamma                $ &: Adyabatisite \\
$ \vec{B}               $ &: Bloch Vektörü \\
$ \vec{\sigma}          $ &: Pauli-spin Matris Vektörü
\end{longtable}
